\chapter{Introdução}
\label{chap:intro} 
\thispagestyle{empty}

A equipe Milhagem UFMG constrói protótipos de veículos para participar de competições
de eficiência energética. Atualmente a equipe possui dois veículos, o M84 e o DT1 que
são respectivamente movidos a gasolina e bateria elétrica. As competições em que a
equipe tem participação são a Shell Eco-Marathon Brasil (nacional) e a Shell EcoMarathon Americas (internacional).
Nessas competições os veículos devem consumir a
menor quantidade de energia para percorrer trajeto, ou seja, devem ter a maior eficiência
energética.

Os principais fatores que influenciam no consumo de energia do veículo são a
aerodinâmica, o peso total, a resistência ao rolamento, o relevo do trajeto e a estratégia
de pista. Essa estratégia consiste na forma e nos momentos em que o motor deve ser
acionado. Um exemplo de uma estratégia de pista muito utilizada nesses protótipos é a
estratégia start-stop, na qual o motor é desligado quando a velocidade é maior que 30
km/h e religado apenas quando é menor que 20 km/h, semelhante a um controle on-off
com histerese.

Durante a avaliação da eficiência energética de um protótipo ele deve seguir as seguintes
restrições: posições inicial e final fixas, velocidade inicial nula, velocidade instantânea
máxima e velocidade média mínima. Uma vez que essas restrições permitem inúmeras
estratégias de pista tem-se a necessidade encontrar a estratégia que maximize a
eficiência energética do veículo durante a avaliação de consumo. Além disso, os protótipos
estão em constante evolução, de forma que essa evolução também modifica as
possibilidades de estratégia de pista o que gera a necessidade de uma solução que
também adaptativa.

As teorias de controle ótimo e estimação de sistemas fornecem o embasamento para a
criação de um software que calcula a estratégia de maior eficiência e adaptando-se a
evolução do protótipo. Neste projeto, o software será constituído de dois estágios, no
primeiro os parâmetros do modelo matemático do veículo serão estimados mediante
dados de um ensaio no protótipo e no segundo estágio a estratégia ótima será calculada
dado o modelo estimado no estágio anterior.

\section{Motivação e Justificativa}
\label{sec:motivacao}

Argumente sobre a importância do projeto desenvolvido usando uma visão de alto nível, sem entrar em detalhes. Contextualize seu projeto dentro do
local de execução ou da literatura e explique como ele é necessário ou inovador. É possível fazer uma breve revisão bibliográfica, confrontando seu
trabalho com outras referências bibliográficas para mostrar a sua contribuição. No quesito contribuição, é muito importante deixar claro o tempo todo
que partes do projetos foram executadas por outros e que partes foram executadas por você. Caso contrário, corre-se o risco de inadvertidademente
tomar crédito pelo trabalho de outrem, o que pode ter implicações legais.

\section{Objetivos do Projeto}
\label{sec:objetivos}

Tendo em vista o exposto acima, este projeto tem por objetivos:

\begin{enumerate}[a)]
    \item Definir um modelo matemático para o protótipo DT1;
    \item Formular um problema de controle ótimo (OCP) pra obter a estratégia ótima;
    \item Implementação de um algoritmo para solução desse OCP;
    \item Definir a estratégia ótima.
\end{enumerate}

\section{Local de Realização}
\label{sec:empresa}

Esse projeto foi desenvolvido na equipe de Milhagem UFMG. Equipe sediada no Departamento de Engenharia Mecânica da UFMG, composta por alunos de graduação em engenharia.
Foi fundada, sobre orientação do
professor Paulo Iscold, em 2005 no Centro de Estudos Aeronáuticos (CEA) do qual fez
parte até 2006.
De 2006 a 2011 o projeto da equipe ficou suspenso retornando as atividades, sobre orientação do professor Fabrício Pujatti, no Centro de Tecnologia
de Mobilidade (CTM).

Hipótrico da equipe de participação em competições:
\begin{itemize}
    \item Maratona Universitária de Eficiência Energética
        \begin{itemize}
            \item  Categoria gasolina
            \begin{itemize}
                \item 2005: 2º Lugar, com a marca de $227,6$ [km/L]
                \item 2006: 1º Lugar, com a marca de $598,9$ [km/L]
                \item 2011: 5º Lugar, com a marca de $199,0$ [km/L]
                \item 2013: 4º Lugar, com a marca de $234,9$ [km/L] 
            \end{itemize}
        \end{itemize}
    \item Shell Eco-marathon Brasil
        \begin{itemize}
            \item  Categoria gasolina
            \begin{itemize}
                \item 2016: 2º Lugar, com a marca de $196,0$ [km/L] 
            \end{itemize}
            \item  Categoria elétrico
            \begin{itemize}
                \item 2017: 3º Lugar, com a marca de $315,6$ [km/kWh]
                \item 2018: 1º Lugar, com a marca de $266,4$ [km/kWh]
            \end{itemize}
        \end{itemize}
    \item Shell Eco-marathon Americas
        \begin{itemize}
            \item  Categoria elétrico
            \begin{itemize}
                \item 2018: 6º Lugar, com a marca de $266,5$ [km/kWh]
                \item 2019: 2º Lugar, com a marca de $226,9$ [km/kWh]
            \end{itemize}
        \end{itemize}
\end{itemize}

\section{Estrutura da Monografia}
\label{sec:organizacao}

O relatório está dividido em quatro capítulos. Este capítulo apresentou uma introdução ao projeto a ser descrito nesta monografia e a equipe onde o
trabalho foi realizado. O Capítulo 2 descreve os princípios básicos de um sistema ... (sistema onde se insere o trabalho) e abrange todos os
conceitos necessários para um melhor entendimento do projeto. O Capítulo 3 aborda a metodologia de desenvolvimento, seguida pela implementação dos
.... No Capítulo 4 tem-se a conclusão da  monografia e algumas sugestões e dificuldades encontradas na realização do projeto.

\clearpage