\addcontentsline{toc}{chapter}{Abstract}

\begin{center}
\huge{{\bf Abstract}}
\vspace{2cm}
\end{center}

The UFMG's electric vehicle, the DT1, is a prototype used in energy efficiency competitions. 
Its most recent autonomy measurement is a $226.9$ [km/kWh] mark held in the 2019 Shell EcoMarathon Americas student competition. 
This end-of-course project aims to determine, based on the application of the optimal control theory, the sequence and activation of the electric motor 
(track strategy) so that the autonomy of the DT1 is the maximum possible on the current track of the Shell EcoMarathon Americas competition at the Sonoma raceway.

To achieve this objective, a mathematical model of the longitudinal dynamics of the prototype vehicle was first determined. 
Based on this model and the determinations of the Shell EcoMarathon organization for measuring autonomy, an optimal control problem was formulated that 
was solved using the FALCON.m software. However, the vehicle model has not been validated experimentally, due to the realization of this project during 
the Emergency Remote Education regime and this does not allow us to state that the track strategy found is optimal for the real case.

\textbf{Keywords} -- Electric Vehicle, Energy Efficiency, Shell EcoMarathon Competition, Vehicle Model, Optimal Control Problem, FALCON.m.

 
\clearpage
\thispagestyle{empty}
\cleardoublepage

