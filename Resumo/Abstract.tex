\addcontentsline{toc}{chapter}{Abstract}

\begin{center}
\huge{{\bf Abstract}}
\vspace{2cm}
\end{center}

The UFMG's electric vehicle, the DT1, is a prototype used in energy efficiency competitions.
The current measurement of your consumption is $226.9$~[km/kWh] performed at the 2019 Shell EcoMarathon Americas student competition.
This end-of-course project aims to determine, based on the application of the optimal control theory, the open loop control signal for motor activation
electric (track strategy) and the transmission ratio so that the consumption of the DT1 is as little as possible on the current track of the Shell EcoMarathon Americas competition at the Sonoma kart track.

To achieve this goal, a mathematical model of the longitudinal dynamics of the prototype vehicle was first determined. Based on this model
and in the determinations of the Shell EcoMarathon organization for an autonomy duration, an optimal control problem was formulated that was solved using the
\textit{software} FALCON.m.

The track strategy and gear ratio found would take the prototype's consumption to $905.7$~[km/kWh]. The prototype model
it was not validated experimentally it was not possible to state that this strategy and this transmission ratio are great for the real case. However it was possible
obtain the following rights: the engine must be started in the most active sections of the track and the total time spent must be the maximum allowed.

\textbf{Keywords} -- Electric Vehicle, Energy Efficiency, Shell EcoMarathon Competition, Vehicle Model, Optimal Control Problem, FALCON.m.

 
\clearpage
\thispagestyle{empty}
\cleardoublepage

