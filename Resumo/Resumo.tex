\addcontentsline{toc}{chapter}{Resumo}

\begin{center}
\huge{{\bf Resumo}}
\vspace{2cm}
\end{center}

% No Resumo, em uma única página, em no máximo dois parágrafos, você explicita os seguintes itens: objetivos do projeto e descrição sucinta do local onde ele foi desenvolvido; metodologia utilizada; e resultados alcançados. Leitores experientes decidem se prosseguirão para a leitura do texto completo após lerem o resumo, a conclusão e a introdução. Por isso nestes lugares você deve colocar um esforço maior de convencimento. Além disso, a linguagem utilizada deve ser acessível a leitores com pouca familiaridade com a área, limitando o uso de jargões.
 
% \begin{sloppypar}
% Este novo parágrafo serve para mostrar que ao pular uma ou mais linhas no texto do arquivo .tex, o \TeX\ entende que você está iniciando outro parágrafo. O comando sloppypar força o texto a não ultrapassar as margens. Só deve ser usado se este problema ocorrer.
% \end{sloppypar}

O veículo elétrico da UFMG, o DT1, é um protótipo utilizado em competições de eficiência energética.
Sua mais recente medição de autonomia é uma marca de $226,9$ [km/kWh] realizada na competição estudantil Shell EcoMarathon Americas de 2019.
Este projeto de fim de curso tem como objetivo determinar, a partir da aplicação da teoria de controle ótimo, a sequencia e de acionamento do motor 
elétrico (estrátegia de pista) para que a autonomia do DT1 seja a máxima possível na atual pista da competição Shell EcoMarathon Americas no kartódromo de Sonoma.

Para realizar este objetivo, primeiramente foi determinado um modelo matemático da dinâmica longitudinal do veículo protótipo. Com base neste modelo
e nas determinações da organização Shell EcoMarathon para a medição da autonomia foi formulado um problema de controle ótimo que foi resolvido utilizando o
\textit{software} FALCON.m. Contudo o modelo do veiculo não foi validado experimentalmente, em razão da realização deste projeto durante o regime de Ensino
Remeto Emergencial e isto não permite afirmar que a estrategia de pista encontrada é ótima para o caso real.
 
\textbf{Palavras-chave} -- Veículo Elétrico, Eficiência Energética, Competição Shell EcoMarathon, Modelo Veícular, Problema de Controle Ótimo, FALCON.m.

\clearpage
\thispagestyle{empty}
\cleardoublepage

