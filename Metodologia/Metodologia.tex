\chapter{Metodologia}
\label{chap:metodologia}
\thispagestyle{empty}

\section{Definição do problema}

\section{Solução Numérica}

\subsection{\textit{FALCON.m}}

\textit{FALCON.m} é um biblioteca de \textit{MATLAB}\textsuperscript{\textregistered}, orientada a objetos, desenvolvida no \textit{Institute of Flight System Dynamics} da \textit{Technische Universit{\"a}t M{\"u}nchen} para resolução e análise de problemas de controle ótimo utilizando \cite{manual:Falcon}.



Na utilização do \textit{FALCON.m} para resolver um problema de controle ótimo são realizadas as seguintes etapas \cite{phd:Matthias}:

\begin{enumerate}
    \item \textbf{Implementar os modelos dinâmicos:} o modelo dinâmico do sistema deve ser implementado em funções do \textit{MATLAB}\textsuperscript{\textregistered}. 
    \item \textbf{Construir os modelos dinâmicos gerais dos subsistemas:} utilizar o e \textit{Model Builder} do \textit{FALCON.m} para relacionar os modelos dinâmicos nas diferentes fases do OCP.
    \item \textbf{Derivar automaticamente todas as funções necessária:} o \textit{Model Builder}irá criar, automaticamente, as derivadas analíticas de todos os subsistemas e combiná-los ao gradiente geral dos modelos usando a regra da cadeia.
    \item \textbf{Gerar código automaticamente para os modelos e compilar:}
    \item \textbf{Implementar funções de restrição adicionais:} repetir os passos de 1 à 4 para todas as restrições e funções de custo.
    \item \textbf{Definir a estrutura do problema de controle ótimo:}
    \item \textbf{Discretizar o problema de controle ótimo:}
    \item \textbf{Resolver o problema discretizado:} o problema de otimização numérica resultante da discretização do problema de controle ótimo é resolvido usando um solucionador numérico apropriado (\textit{IPOPT}, \textit{SNOPT}, \textit{FMINCON} ou \textit{WORHP}).
    \item \textbf{Analisar dos resultados:}
\end{enumerate}

Detalhes sobre o uso do \textit{FALCON.m} pode ser encontrados em sua documentação \cite{manual:Falcon}, bem com uma explicação detalhada da sua implementação na tese de doutorado de \citeauthor{phd:Rieck}.


Esse é um teste de como escreve codigo  \lstinline[style=Matlab-editor]{Bake()}

\clearpage