\chapter{Conclusões}
\label{chap:conclusao}
\thispagestyle{empty}

O capítulo de conclusão deste projeto de fim de curso está divido em duas seções.
Na Seção \ref{sec:5_1} são feitas as considerações finais sobre os resultados alcançados e uma avaliação do cumprimento dos objetivos definidos inicialmente no Capítulo \ref{chap:intro}.
As propostas de continuidade para o projeto estão na Seção \ref{sec:5_2}.

\section{Considerações Finais}
\label{sec:5_1}

Os objetivos (a) e (d) de definir o modelo matemático para a dinâmica do protótipo DT1 e de definir a estratégia ótima para a pista da Shell EcoMarathon Américas de 2019 
foram atingidos de forma parcial uma vez que, como discutido na Seção \ref{sec:resultados_modelo}, não foi realizado uma validação experimental do modelo e, desta forma, não é possível afirmar que o modelo determinado representa fidedignamente o comportamento do veículo
e por consequência que a estratégia de pista e a relação de transmissão encontradas são ótimas para o caso real. Contudo foi possível obter as seguintes conclusões a partir da estrategia encontrada que 
podem ser utilizadas como orientações na realização da tentativa: 

\begin{itemize}
    \item O motor deve ser ligado no trecho de maior aclive da pista;
    \item O tempo total gasto deve ser o máximo permitido;
    \item O tempo da ultima volta é maior que o tempo das demais;
    \item A velocidade final é maior que zero.
\end{itemize}



Formular o problema de controle ótimo pra obter a estratégia ótima e implementar o algoritmo para solução desse OCP,
que são objetos (b) e (c), foram cumpridos integralmente pois a formulação contempla todas as restrições para o comportamento do veículo durante uma tentativa e o solução cumpriu todas estas
restrições com um tempo de computação pequeno. Além disso a formulação do OCP e o código serão de fácil modificação nas propostas de continuidade do trabalho
a fim de concluir os objetivos (a) e (d).

\section{Propostas de Continuidade}
\label{sec:5_2}

A principal proposta para continuidade deste projeto e evolução e validação experimental do modelo matemático do protótipo DT1.
Realizar a validação experimental é de fundamental importância para que os resultados obtidos com a solução do OCP possam ser utilizados pela equipe Milhagem UFMG.
Essa evolução deve considerar pontos como a resistência ao rolamento do pneu em curvas, o atrito viscoso dos rolamento e perdas na bateria e no conversor de potência.

Outro ponto de continuidade é a modificação do código escrito pra que a dinâmica e perdas do indutor da armadura motor sejam consideradas a fim de confirmar
se são desconsideráveis.  
Modificar, também, a restrição de caminho da corrente de forma que ela possa atingir valores maiores que o limite de $30$ [A] durante a partida para que 
seja avaliada na otimização a possibilidade de usar a corrente de pico do motor, que é de $63$ [A], para que o protótipo tenha
uma maior aceleração no momento da partida. 

\clearpage
